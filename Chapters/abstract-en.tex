%!TEX root = ../template.tex
%%%%%%%%%%%%%%%%%%%%%%%%%%%%%%%%%%%%%%%%%%%%%%%%%%%%%%%%%%%%%%%%%%%%
%% abstract-en.tex
%% NOVA thesis document file
%%
%% Abstract in English([^%]*)
%%%%%%%%%%%%%%%%%%%%%%%%%%%%%%%%%%%%%%%%%%%%%%%%%%%%%%%%%%%%%%%%%%%%

\typeout{NT FILE abstract-en.tex}%

Ethical issues in software engineering are becoming more important, however incorporating them into requirements engineering (RE) is still 
difficult. There are several methods and tools to help identify and discuss ethical requirements, although these approaches often focus in 
facilitating discussions rather than providing structured and automated processes. In addition, many of the existing methods tend to focus on 
specific situations, which makes them less flexible when it comes to addressing the evolving ethical concerns that can arise in different projects. 
This thesis aims to explore how we can systematically gather and define ethical requirements. By using a catalog-based approach, we offer a 
structured methodology for integrating ethical considerations into standard RE practices. Additionally, it looks into the potential role of large 
language models (LLMs) in automating parts of ethical requirement generation, consequently improving efficiency while maintaining ethical awareness.


\keywords{
  Ethical Requirements \and
  Requirements Engineering \and
  Software Ethics \and
  Large Language Models
}
