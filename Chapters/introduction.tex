%!TEX root = ../template.tex
%%%%%%%%%%%%%%%%%%%%%%%%%%%%%%%%%%%%%%%%%%%%%%%%%%%%%%%%%%%%%%%%%%%
%% chapter1.tex
%% NOVA thesis document file
%%
%% Chapter with introduction
%%%%%%%%%%%%%%%%%%%%%%%%%%%%%%%%%%%%%%%%%%%%%%%%%%%%%%%%%%%%%%%%%%%

\typeout{NT FILE introduction.tex}%

\chapter{Introduction}
\label{cha:introduction}

\prependtographicspath{{Chapters/Figures/Covers/}}

% epigraph configuration
\epigraphfontsize{\small\itshape}
\setlength\epigraphwidth{12.5cm}
\setlength\epigraphrule{0pt}


\section{Context and Motivation}
With the rapid advancements in technology, there is an increasing need to integrate ethical standards into software products \cite{gogoll2021ethics}. Numerous controversial cases have emerged 
regarding ethics in the tech industry, particularly concerning the collection of personal data without consent — for example, South Korea’s privacy watchdog recently fined Meta \$15 million for 
unlawfully collecting sensitive personal data\cite{ReutersMeta2024}, and investigations in Australia have revealed that personal information is being shared repeatedly via ad‐tracking systems without user consent\cite{AussiesData2024}. 

Currently, the emphasis on ethical requirements is primarily directed at Artificial Intelligence (AI) systems, as the decisions made by these systems can significantly impact society; recent international efforts - including
a legally binding treaty between the US, UK, and EU on AI standards - underscore this focus \cite{TETAIStandards2024}. However, there is still a lack of research on how to effectively incorporate 
ethical requirements into software products in general, with most studies focusing on ethics for AI systems than the broader software development process \cite{zuber2022empowered}.

Recent advancements in large language models (LLMs) for reading and interpreting human text \cite{lu2023emergent} present an opportunity to more effectively extract stakeholder needs. 
We intend to leverage these models to improve the extraction of ethical requirements from stakeholders' textual data.

\section{Problem Description}
The lack of an efficient approach to identifying ethical requirements for software products presents a significant challenge. The rapid advancement of technology has led to a wide range of 
ethical issues that must be considered when designing software \cite{gogoll2021ethics}. 

Ethical requirements are often implicit or abstract, making them difficult to identify and clearly define \cite{biableethics2022}. Furthermore, there is often a communication barrier between 
stakeholders—their understanding of ethical issues may vary due to differences in background, culture, and personal beliefs—and the developer teams responsible for translating these concerns into 
software requirements \cite{biableethics2022}.
Additionally, different stakeholders may hold varying perspectives, resulting in conflicting priorities. Finally, converting abstract ethical principles into specific, testable 
system requirements is complex and resource-intensive, often straining project timelines and available resources \cite{gogoll2021ethics}.

\section{Objectives}
Our objective is to develop a software tool that receives textual data from stakeholders, including detailed representations of their software product concepts. This tool will extract ethical 
requirements and generate a comprehensive specification of Ethical User Stories that effectively address these requirements.

The software will leverage large language models (LLMs) to analyze and interpret the input provided by stakeholders. Additionally, it will simulate human behavior, allowing for a deeper insight
into potential ethical dilemmas and social implication associated with the project. This will empower teams to develop software products that align with ethical standards and social expectations.



\section{Document Structure}
This section outlines the structure of the remaining chapters in this thesis.

\begin{enumerate}
    \item \textbf{Chapter 2} provides an overview of essential concepts and technologies pertinent to the study, including requirements engineering, large language models, and ethics.
    \item \textbf{Chapter 3} presents the state of the art reasearch conducted to identify and evaluate existing approaches to gathering requirements and their limitations.
    \item \textbf{Chapter 4} discusses the exprected contributions of this thesis and our proposed tool.
\end{enumerate}

